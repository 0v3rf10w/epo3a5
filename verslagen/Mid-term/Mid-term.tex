\documentclass[oneside,dutch]{tudelft-report}
\DeclareGraphicsExtensions{.pdf,.png,.jpg,.jpeg}
%\usepackage[numbered, framed]{mcode}
\usepackage{subcaption}
\usepackage{float}
\usepackage[dutch]{babel}
\usepackage{calc}  
\usepackage{enumitem}  
\usepackage{tabto}
\usepackage{adjustbox}
\usepackage{blindtext}

\begin{document}

\frontmatter

\title{Mid-term report}
\author{Projectgroep A5}
\affiliation{TU Delft}
\maketitle
\chapter{Samenvatting}


\chapter{Inleiding}
\newpage

\chapter{Probleemstelling}
\newpage

\chapter{Systeem overzicht}
\newpage

\chapter{Besturing}
Ons systeem wordt gespeeld doormiddel van 2 soorten besturing. De Ultrasone en de Button. De speler kan de modus selecteren doormiddel van een switch. Deze data van de buttons en de ultrasone worden verwerkt door een Arduino. Er is gekozen voor deze optie omdat de Arduino via SPI werkt. Hierdoor kan de SPI code getest worden en kunnen we het aantal pinnen dat gebruikt wordt laag houden. 
\newpage
\chapter{Blackbox}
\newpage

\chapter{ALU/PC}
\newpage

\chapter{VGA}
\newpage

\chapter{SD-kaart/SPI}
\newpage

\chapter{Game: Pong}

\end{document}